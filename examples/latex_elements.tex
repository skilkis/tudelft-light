\chapter{Example \LaTeX\ Elements}
\pagenumbering{arabic} %DO NOT REMOVE FROM THE INTRODUCTION CHAPTER
This template has been developed for the [AE3200] Design Synthesis Exercise by
\c{S}an K{\i}lk{\i}\c{s} and Munyung Kim. The source code can be modified and
redistributed but the license file must remain intact. Refer to the
\texttt{LICENSE.md} included with the template for details.

\section{Tables \& Figures}
An example \cref{tab:exampletable} and an example \cref{fig:flame} can be found
in this section. When you label tables or figures, make sure to use `tab:name'
or `fig:name', this is not necessary for syntax but makes organization and
look-up of labels easier. For inserting 2+ figures in a row, look at the
formatting of \cref{fig:sbs}. Using the \texttt{cleveref} package
negates the need for manually typing `Table' or `Figure'. The syntax is as
follows, note that the `tab' in `tab:exampletable' is not necessary for
\texttt{cref} and is purely for organizational reasons. However a `,' cannot be
utilized as this is interpreted as a list.

\begin{verbatim}
    \cref{tab:exampletable}
\end{verbatim}

The Tables below use the package \texttt{tabularx} which adjusts column spacing
automatically to fit the table within the margins of the page. The syntax is as
follows where 'L' is for Left Aligned, 'C' for Centered, and 'R' is for Right
Aligned:

\begin{verbatim}
    \begin{tabularx}{\textwidth}{L C C C}
\end{verbatim}

In order to keep up the same appearance for all tables use the commands
\texttt{toprule}, \texttt{midrule}, \texttt{bottomrule}, and \texttt{hdashline}
to create the horizontal lines. NO VERTICAL LINES ARE ALLOWED!

\begin{table}[H]
	\centering
	\caption{Example Table}
	\label{tab:exampletable}
	\begin{tabularx}{\textwidth}{L C C C} %'L' for Left Aligned, 'C' for Centered, 'R' for Right Aligned
	    \toprule\
		\textbf{Component}				& \textbf{Mass [$\text{kg}$]}	&\textbf{Location [$\text{m}$]} & \textbf{Location [\% MAC]}   \\ \toprule
		Wing 							& 425.4 						& 5.74 							& 40.00\\\hdashline
		Main Landing Gear 				& 243.1 						& 5.82 							& 45.00 \\\hdashline
		Fuel System 					& 80.74 						& 5.91 							& 50.00 \\\hdashline
		Flight Control System 			& 48.61 						& 6.08 							& 60.00	\\\hdashline
		Hydraulics 						& 4.660 						& 6.08 						    & 60.00 \\\hdashline
		\textbf{Wing Group} 			& \textbf{802.5} 				& \textbf{5.80} 				& \textbf{43.85}\\ \midrule
		Fuselage 						& 265.2 						& 5.74 	                        & 40.00 \\\hdashline
		Engine 							& 409.4 						& 1.64 							& - \\\hdashline
		Avionics 						& 490.9 						& 4.39 							& - \\\hdashline
		H. Tail 						& 42.93 						& 13.2 						    & - \\\hdashline
		V. Tail 						& 66.43 						& 12.6 						    & - \\\hdashline
		Nose Gear						& 54.58 						& 2.50 							& - \\\hdashline
		Electrical 						& 217.4 						& 6.16 							& 67.12 \\\hdashline
		AC \& Anti-Ice 					& 215.7 						& 6.16 							& 67.12 \\\hdashline
		Furnishings 					& 241.5 						& 6.16 							& 67.12 \\\hdashline
		\textbf{Fuselage Group} 		& \textbf{2004} 				& \textbf{5.01} 				& \textbf{-2.32} \\ \midrule
		\textbf{OEW C.G.} 				& \textbf{2806} 				& \textbf{5.24} 				& \textbf{10.88} \\ \bottomrule
	\end{tabularx}
\end{table}

\begin{table}[H]
	\centering
	\caption{Example Table II}
	\label{tab:exampletableII}
    \begin{tabularx}{\textwidth}{C C C C C C C C} %'L' for Left Aligned, 'C' for Centered, 'R' for Right Aligned
    \toprule\
    {$m$} & {$\Re\{\underline{\mathfrak{X}}(m)\}$} & {$-\Im\{\underline{\mathfrak{X}}(m)\}$} & {$\mathfrak{X}(m)$} & {$\frac{\mathfrak{X}(m)}{23}$} & {$A_m$} & {$\varphi(m)\ /\ ^{\circ}$} & {$\varphi_m\ /\ ^{\circ}$} \\ \toprule
    1  & 16.128 & +8.872 & 16.128 & 1.402 & 1.373 & -146.6 & -137.6 \\ \hdashline
    2  & 3.442  & -2.509 & 3.442  & 0.299 & 0.343 & 133.2  & 152.4  \\ \hdashline
    3  & 1.826  & -0.363 & 1.826  & 0.159 & 0.119 & 168.5  & -161.1 \\ \hdashline
    4  & 0.993  & -0.429 & 0.993  & 0.086 & 0.08  & 25.6   & 90     \\ \midrule
    5  & 1.29   & +0.099 & 1.29   & 0.112 & 0.097 & -175.6 & -114.7 \\ \hdashline
    6  & 0.483  & -0.183 & 0.483  & 0.042 & 0.063 & 22.3   & 122.5  \\ \hdashline
    7  & 0.766  & -0.475 & 0.766  & 0.067 & 0.039 & 141.6  & -122   \\ \hdashline
    8  & 0.624  & +0.365 & 0.624  & 0.054 & 0.04  & -35.7  & 90     \\ \midrule
    9  & 0.641  & -0.466 & 0.641  & 0.056 & 0.045 & 133.3  & -106.3 \\ \hdashline
    10 & 0.45   & +0.421 & 0.45   & 0.039 & 0.034 & -69.4  & 110.9  \\ \hdashline
    11 & 0.598  & -0.597 & 0.598  & 0.052 & 0.025 & 92.3   & -109.3 \\ \bottomrule
    \end{tabularx}
\end{table}

\begin{figure}[H]
    \centering
    \includegraphics[width=0.3\textwidth]{examples/static/flame.jpg}
    \caption{TU Delft Logo Flame}
    \label{fig:flame}
\end{figure}

\begin{figure}[H]
\centering
\begin{subfigure}[b]{0.5\textwidth}
  \centering
  \includegraphics[width=.85\textwidth]{examples/static/flame.jpg}
  \subcaption{TU Delft Logo Flame}
  \label{fig:flame1}
\end{subfigure}%
\begin{subfigure}[b]{0.5\textwidth}
  \centering
  \includegraphics[width=.85\textwidth]{examples/static/flame.jpg}
  \subcaption{TU Delft Logo Flame}
  \label{fig:flame2}
\end{subfigure}
\caption{Two Figures Side-by-Side} %Main Caption
\label{fig:sbs}
\end{figure}


\section{References \& Citations}
The \texttt{biblatex} package is used for references with the default `numeric'
style for in-text citations and references \cite{sampleref}. The references
sorting style is set to `none' meaning that the references are sorted by the
order in which they appear in text. A sample file \texttt{samplerefs.bib} is
included to help when dealing with different types of publications.

\begin{verbatim}
    \cite{citationtag}
\end{verbatim}

% TODO add syntax for adding page numbers
% TODO add 

\section{Equations \& Nomenclature}
When typesetting equations, you need to use a nomenclature code when you
introduce a variable for the FIRST time, such that the variable is listed on
the list of symbols. An example is given below by \cref{eq:exampleeq}. With the
current implementation, duplicate nomenclature items are not automatically
removed.

\begin{equation}
\label{eq:exampleeq}
    L = \frac{1}{2}\rho V^2 S \cdot C_{L}
\end{equation}

\nomenclature[A]{ABCD}{Ayy Bee See Dee}
\nomenclature[B]{$C_L$}{Lift Coefficient \nomunit{-}}
\nomenclature[B, 01]{$V$}{Velocity \nomunit{kg.m^{-1}}}
\nomenclature[B, 02]{$S$}{Wing Area \nomunit{m^{2}}}
\nomenclature[G]{$\rho$}{Density of Air \nomunit{kg.m^{-3}}}

The the list of symbols for the above equation were generated with the code
below:

\begin{verbatim}
    \nomenclature[A]{ABCD}{Ayy Bee See Dee}
    \nomenclature[B]{$C_L$}{Lift Coefficient \nomunit{-}}
    \nomenclature[B, 01]{$V$}{Velocity \nomunit{kg.m^{-1}}}
    \nomenclature[B, 02]{$S$}{Wing Area \nomunit{m^{2}}}
    \nomenclature[G]{$\rho$}{Density of Air \nomunit{kg.m^{-3}}}
\end{verbatim}


\section{Units and Numbers}
To have uniform spacing and formatting of numbers and units the
\textsf{siunitx} package can be used. The syntax for displaying a number
with its corresponding unit as ``\SI{5}{\kilogram}'' is as follows:

\begin{verbatim}
    \SI{5}{\kilogram}
\end{verbatim}

Formatting of a unit of measure as ``\si{\kilogram}'' is as follows, pay
close attention to the lower-case call to \texttt{\textbackslash si}.

\begin{verbatim}
    \si{\kilogram}
\end{verbatim}

\section{Research Questions}
Research questions can be formatted using the \texttt{questions} environment
as follows:

\begin{verbatim}
    \begin{questions}
        \item \label{rq:meaning} What is the meaning of life?
        \begin{questions}
            \item What is the answer to the Ultimate Question of Life, the 
            Universe, and Everything?
        \end{questions}
    \end{questions}
\end{verbatim}

This produces the following output:

\begin{questions}
    \item \label{rq:meaning} What is the meaning of life?
    \begin{questions}
        \item What is the answer to the Ultimate Question of Life, the 
        Universe, and Everything?
    \end{questions}
\end{questions}


Note that individual questions can be referenced using the declared label, for
example by using \texttt{\textbackslash cref\{rq:meaning\}}. Doing so would
render the reference as \cref{rq:meaning}.

% TODO show some examples for code listings